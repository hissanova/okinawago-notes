\documentclass[a4j,12pt]{jarticle}
\usepackage{okumacro}

% \usepackage{amsthm}
%\usepackage{color}

\usepackage{ntheorem}
\theorembodyfont{\upshape}
\theoremstyle{break}
\newtheorem{example}{例文}


\newcommand{\highlight}[1]{
  \textbf{\underline{#1}}
}
\begin{document}
\section*{テーン形例文集}
\begin{example}
  \label{ex:p2}
  \begin{itemize}
  \item p.2
  \item 時制:現在
  \item 相:結果継続(トーン形)
  \item モード:意外性(テーン形)
  \item (沖普)あき あき、\ruby{何}{ぬー}が。くぬ\ruby{着物}{ちのー}\highlight{\ruby{破}{や}りとーてーさ}。
  \item (日)あれ、あれ、どうしたんだ。この着物は破れていたんだ。
  \item (沖丁)あき あき、\ruby{何}{ぬー}やいびーが。くぬ\ruby{着物}{ちのー}\highlight{\ruby{破}{や}りとーいびーてーさ}。
\end{itemize}
\end{example}
\begin{example}
  \begin{itemize}
    \item p.3
    \item 時制:現在
    \item 相:パーフェクト(テーン形)
    \item (沖普)あゐっあゐっ、ちゃーる\ruby{積}{ちむ}ゐが、あんし\ruby{捨}{し}てぃほーりー\highlight{しぇー}しが。
    \item (日)あれっあれっ、どういうつもりかね。そんなに捨て散らかっているけど。
    \item (沖普)あゐっあゐっ、ちゃーる\ruby{積}{ちむ}ゐやいびーが、あんし\ruby{捨}{し}てぃほーりー\highlight{しぇーいびー}しが。
  \end{itemize}
\end{example}
\begin{example}
  \begin{itemize}
    \item p.6
    \item 時制:現在
    \item モード:意外性(テーン形)
    \item (沖普)うねうね、あんし\ruby{上手}{じょうじ}\highlight{やてーる}。なー\ruby{大事}{でーじ}なむんやっさー。
    \item (日)おやおや、そんなに上手だったんだね。大したもんだよ。
    \item (丁寧)うねうね、あんし\ruby{上手}{じょうじ}\highlight{やいびーてーる}。なー\ruby{大事}{でーじ}なむんやいびーっさー。
    \item (尊敬)うねうね、あんし\ruby{上手}{じょうじ}\highlight{やみしぇーてーる}。なー\ruby{大事}{でーじ}なむんやいびーっさー。
    \item (尊敬+丁寧)うねうね、あんし\ruby{上手}{じょうじ}\highlight{やみしぇーびーてーる}。なー\ruby{大事}{でーじ}なむんやいびーっさー。
  \end{itemize}
\end{example}
\begin{example}
  \begin{itemize}
  \item p.12
  \item 時制:現在
  \item モード:パーフェクト>間接確認(テーン形)(?)
  \item (沖普)\ruby{髪}{からじ}ぬもーゐくゎんくゎん そーしが、\ruby{今}{なま}どぅ\highlight{\ruby{起}{う}きてーさやー}。
  \item (日)髪がぐちゃぐちゃになっているが、今起きたんだね。
  \item (沖丁)\ruby{髪}{からじ}ぬもーゐくゎんくゎん そーいびーしが、\ruby{今}{なま}どぅ\highlight{\ruby{起}{う}きてーいびーさやー}。
  \end{itemize}
\end{example}
\begin{example}
  \begin{itemize}
  \item p.14
  \item 時制:現在
  \item 相:パーフェクト(テーン形)
  \item (沖普)くぬ\ruby{着物}{ちん}よー、\ruby{昨日}{ちぬー}\highlight{\ruby{買}{こー}てぃちぇー}しが、\ruby{似合}{うちゃ}とーみ。
  \item (日)この着物はね、昨日買ってきたんだけど似合っている。
  \item (沖丁)くぬ\ruby{着物}{ちん}よーさい、\ruby{昨日}{ちぬー}\highlight{\ruby{買}{こー}てぃちぇーいびー}しが、\ruby{似合}{うちゃ}とーいびーみ。
  \end{itemize}
\end{example}
\begin{example}
  \begin{itemize}
    \item p.17
    \item 時制:現在
    \item モード:意外性(テーン形)
    \item (沖普)けーてー\ruby{壊}{やん}てぃどぅ あるむんなー、あんし\ruby{不器用}{ぶくー}\highlight{やてーる}。
    \item (日)かえって壊してあるんじゃないの、そんなに不器用だったとはね。
    \item (丁寧)けーてー\ruby{壊}{やん}てぃどぅ あいびーるむんなー、あんし\ruby{不器用}{ぶくー}\highlight{やいびーてーる}。
    \item (尊敬)けーてー\ruby{壊}{やん}てぃどぅ あいびーるむんなー、あんし\ruby{不器用}{ぶくー}\highlight{やみしぇーてーる}。
    \item (尊敬+丁寧)けーてー\ruby{壊}{やん}てぃどぅ あいびーるむんなー、あんし\ruby{不器用}{ぶくー}\highlight{やみしぇーびーてーる}。
  \end{itemize}
\end{example}
\begin{example}
  \begin{itemize}
  \item p.20
  \item 時制:現在
  \item 相:パーフェクト?継続相?(テーン形)
  \item (沖普)\ruby{先}{さち}なゐ\ruby{競争}{すーぶ}やあらんどー、ちゅらーく\highlight{\ruby{完成}{とぅじ}みてーし}が \ruby{一番}{いちばん}やさ。
  \item (日)先になれば良いと言う事ではないよ、きちんと完結させたのが一番なんだ。
  \item (沖丁)\ruby{先}{さち}なゐ\ruby{競争}{すーぶ}やあいびらんどー、ちゅらーく\highlight{\ruby{完成}{とぅじ}みてーし}が \ruby{一番}{いちばん}やいびーさ。
  \end{itemize}
\end{example}
\begin{example}
  \begin{itemize}
  \item p.21
  \item 時制:現在
  \item 相:継続相(テーン形)
  \item (沖普)\ruby{白湯上戸}{さーゆーじょーぐ}なてぃ、\ruby{常時}{とぅーち}、\ruby{湯}{ゆー}や\ruby{沸}{わ}かち\ruby{冷}{さ}まち\highlight{\ruby{置}{う}ちきてーん}。
  \item (日)白湯が大好きで、常時、湯を沸かし冷まして置いてある。
  \item (沖丁)\ruby{白湯上戸}{さーゆーじょーぐ}なてぃ、\ruby{常時}{とぅーち}、\ruby{湯}{ゆー}や\ruby{沸}{わ}かち\ruby{冷}{さ}まち\highlight{\ruby{置}{う}ちきてーいびーん}。
  \end{itemize}
\end{example}
\begin{example}
  \begin{itemize}
  \item p.24
  \item 時制:現在
  \item 相:パーフェクト(テーン形)
  \item (沖普)\ruby{世界}{しけー}、\ruby{大動}{うーむさげーゐ}\highlight{しみてーる}\ruby{憎}{や}な\ruby{感染病}{ふーちやんめー}コロナや、\ruby{早}{ふぇー}く\ruby{駆除}{とぅゐぬき}らんがやー。
  \item (日)世界を震撼させている憎き感染病を一日も早く駆除できないものかね。
  \item (沖丁)\ruby{世界}{しけー}、\ruby{大動}{うーむさげーゐ}\highlight{しみてーる}\ruby{憎}{や}な\ruby{感染病}{ふーちやんめー}コロナや、\ruby{早}{ふぇー}く\ruby{駆除}{とぅゐぬき}やびらんがやー。
  \end{itemize}
\end{example}
\begin{example}
  \begin{itemize}
  \item p.26
  \item 時制:現在
  \item 相:パーフェクト(テーン形)
  \item (沖普)\ruby{御代}{せーしぬ}ん\ruby{入}{い}りらしよー、くれー\ruby{丹念}{てぃーあんだ}\highlight{\ruby{込}{く}みてーくとぅ} まーさんどー。
  \item (日)御代わりも、させてよ。これは丹念(真心を込めて)に作ったから美味しいよ。
  \item (沖丁)\ruby{御代}{せーしぬ}ん\ruby{入}{い}ってぃくぃみしぇーびりよー、くれー\ruby{丹念}{てぃーあんだ}\highlight{\ruby{込}{く}みてーくとぅ}(\highlight{\ruby{込}{く}みてーいびーくとぅ}?) まーさいびーんどー。
  \end{itemize}
\end{example}
\begin{example}
  \begin{itemize}
  \item p.27
  \item 時制:現在
  \item 相:パーフェクト(テーン形)
  \item (沖普)\ruby{本物}{そーむん}やんでぃち\ruby{買}{こー}らさったるむのー、\highlight{\ruby{模造}{にし}てーし} どぅやたる。
  \item (日)本物だよと売り付けられたが、模造品だった。
  \end{itemize}
\end{example}
\begin{example}
  \begin{itemize}
  \item p.36
  \item 時制:現在
  \item 相:パーフェクト(テーン形)
  \item (沖普)\ruby{名前}{なー}\ruby{負}{ま}きそーさ、\ruby{畏}{うす}りん\ruby{知}{し}らん\highlight{\ruby{名前付}{なーつぃ}きてーさ}。 
  \item (日)名前負けしているよ、身の程を知らない名前を付けてあるよ。
  \item (沖丁)\ruby{名前}{なー}\ruby{負}{ま}きそーいびーさ、\ruby{畏}{うす}りん\ruby{知}{し}らん\highlight{\ruby{名前付}{なーつぃ}きてーいびーさ}。 
  \end{itemize}
\end{example}
\begin{example}
  \begin{itemize}
  \item p.36
  \item 時制:現在
  \item 相:パーフェクト(テーン形)
  \item (沖普)ナーベーラーん\highlight{\ruby{組}{か}ちぇー}くとぅ、\ruby{後}{あとー}\ruby{実}{な}ゐし\ruby{待}{ま}てーしぬさ。
  \item (日)ヘチマ(糸瓜)の棚も組み立てたから、後は実の生るのを待つだけだよ。
  \item (沖丁)ナーベーラーん\highlight{\ruby{組}{か}ちぇー}くとぅ(\highlight{\ruby{組}{か}ちぇーいびー}くとぅ?)、\ruby{後}{あとー}\ruby{実}{な}ゐし\ruby{待}{ま}てーしなびーさ。
  \end{itemize}
\end{example}

\begin{example}
  \begin{itemize}
  \item p.41
  \item 時制:現在
  \item 相:パーフェクト?回想?(形容詞のテーン形?)
  \item (沖普)\ruby{声帯}{ぬーでぃーじる}ぬ\ruby{痛}{や}ぬっさー、\ruby{昨夜}{ゆーべー}カラオケ\ruby{励}{はま}ゐ\highlight{じゅーさてーさ}。
  \item (日)声帯が痛いなあ、昨夜はカラオケを頑張り過ぎたかな。
  \item (沖丁)\ruby{声帯}{ぬーでぃーじる}ぬ\ruby{痛}{や}ぬっさー、\ruby{昨夜}{ゆーべー}カラオケ\ruby{励}{はま}ゐ\highlight{じゅーさいびーてーさ}。
  \end{itemize}
\end{example}
\begin{example}
  \begin{itemize}
  \item p.42
  \item 時制:現在
  \item 相:パーフェクト
  \item (沖普)ねーまゐかかてんぃ\highlight{\ruby{買}{こー}てーし}、たでーまなかゐ けーやんてぃなー。
  \item (日)あんなに、せがんで買った物をこんなに早く壊してしまうなんて。
  \item (沖丁)ねーまゐかかてんぃ\highlight{\ruby{買}{こー}てーし}、たでーまなかゐ けーやんじゃびてぃなー。
  \item (過去形との比較 p.52)\ruby{良}{ま}し\ruby{良}{ま}しさってぃ\highlight{\ruby{買}{こー}たるむの一}、\ruby{粗悪品}{そーべーむん}つぃかまさってぃねーらんさ。
  \item 良いですよ、良いですよと言われて買ったけど粗悪品つかまされたよ。
  \end{itemize}
\end{example}
\begin{example}
  \begin{itemize}
  \item p.42
  \item 時制:現在
  \item モード:意外性(テーン形)
  \item (沖普)\ruby{無}{ねー}ん\ruby{無}{ねー}んすたる\ruby{着物}{ちのー}、うっさ\highlight{あてーる}。
  \item (日)無い無いよと言っていた着物は、かなり持っているじゃないの。
  \item (沖丁)\ruby{無}{ねー}ん\ruby{無}{ねー}んすたる\ruby{着物}{ちのー}、うっさ\highlight{あいびーてーる}。
  \end{itemize}
\end{example}
\begin{example}
  \begin{itemize}
  \item p.53
  \item 時制:現在
  \item 相:パーフェクト
  \item (沖普)\ruby{見}{みー}ぶさ\ruby{物}{むん}ぬ\ruby{恐}{うとぅる}し\ruby{物}{むん}んでぃち、\ruby{先人達}{んかしんちょー}ゆー\highlight{\ruby{言}{い}ちぇーみしぇーさ}。
  \item (日)「怖い物見たさ」と先人達は、的を射る言葉をのこされましたね。
  \item (沖丁)\ruby{見}{みー}ぶさ\ruby{物}{むん}ぬ\ruby{恐}{うとぅる}し\ruby{物}{むん}んでぃち、\ruby{先人達}{んかしんちょー}ゆー\highlight{\ruby{言}{い}ちぇーみしぇーびーさ}。
  \end{itemize}
\end{example}
\begin{example}
  \begin{itemize}
  \item p.64
  \item 時制:現在
  \item 相:パーフェクト
  \item (沖普)\ruby{渋面}{わじゃ}でぃ\ruby{此所}{くま} \ruby{向}{む}かとーしが、\ruby{何}{ぬー}がな \highlight{しーやんてーん}てー。
  \item (日)しかめっ面して此所をむいているが、何か仕損じたのじゃないの。
  \item (沖丁)\ruby{渋面}{わじゃ}でぃ\ruby{此所}{くま} \ruby{向}{む}かとーしが、\ruby{何}{ぬー}がな \highlight{しーやんてーいびーん}てー。
  \end{itemize}
\end{example}
\begin{example}
  \begin{itemize}
  \item p.67
  \item 時制:現在
  \item 相:パーフェクト
  \item (沖普) んちゃ、あんやさ、いゃーむんぬどぅ\highlight{\ruby{正解}{あた}とーてーる}むんなー。
  \item (日) なるほど、そうだね、君のが正解だったんだね。
  \item (沖丁)  んちゃ、あんやさ、いゃーむんぬどぅ\highlight{\ruby{正解}{あた}とーいびーてーる}むんなー。
  \item (沖丁)  んちゃ、あんやさ、いゃーむんぬどぅ\highlight{\ruby{正解}{あた}とーてーいびーる}むんなー。
  \end{itemize}
\end{example}

\section{ネー条件形}
\begin{example}
  \begin{itemize}
  \item p.22
  \item 述語:動詞・オーン形
  \item ニュアンス:仮定が進行中の出来事・動作の時。「〜している時は」。
  \item (沖普) \ruby{思案中}{むぬかんげー}そーゐねー、すっとぅぐぁー そーてぃとぅらしよー。
  \item (日) 思案中には、そっとしておいてね。
  \item (例) むぬかどーいねー:食べている時は
  \end{itemize}
\end{example}
\begin{example}
  \begin{itemize}
  \item p.24
  \item 述語:動詞・オーチュン形
  \item ニュアンス:反実仮想?(まだしていない動作について使うとの事)
  \item (沖普) \ruby{塩幸物}{すーじゅーむん}びけーん\ruby{食}{か}どーちーねー、\ruby{高血圧}{ちーだかー}なゐんどー。
  \item (日) 塩辛い物ばかり食べていると、高血圧になるぞ。
  \end{itemize}
\end{example}

\section{その他のフレーズ}
\begin{example}
  \begin{itemize}
  \item p.68
  \item フレーズ: V$_{過去}$ + てーまん
  \item \ruby{害苦}{んぱんぱ}さんてーまん、うりがー\ruby{逃}{ぬ}がりらんさ、\ruby{諦}{やす}んじてぃ\ruby{早}{ふぇー}く\ruby{通}{とぅー}れー。
  \item 否々しても、それからは逃げられないよ、諦めて早く行きなさい。
  \end{itemize}
\end{example}
\begin{example}
  \begin{itemize}
  \item p.35
  \item フレーズ: V$_{過去}$ + どー(ん)でー
  \item \ruby{滞}{とぅどぅくー}ゆんどー、くぬ\ruby{遅}{にっか}から うっさなー\ruby{食}{か}だんどーでー。
  \item 胃もたれするぞ、こんな遅くから、そんなに食べたんでは。
  \end{itemize}
\end{example}
\begin{example}
  \begin{itemize}
  \item p.36
  \item フレーズ: V$_{過去}$ + どー(ん)でー
  \item \ruby{各自各様}{なーめーめー}、\ruby{勝手勝手}{かってぃかってぃ}に\ruby{発言}{あび}たんどーんでー、くぬ\ruby{話}{はな}しぇーまとぅまらんしが。
  \item 各自各々が勝手に発言したんでは、この話はまとまらないよ。
  \end{itemize}
\end{example}
\begin{example}
  \begin{itemize}
  \item p.36
  \item フレーズ: V$_{連体}$ + むん
  \item なー、くぬ\ruby{遅}{にっか}なとーるむんなー、でぃー\ruby{今日}{ちゅー}やくりっし\ruby{済}{すぃ}まさな。
  \item もう、こんなに遅くなっているよ。さあ、今日はこれで切り上げようや。
  \end{itemize}
\end{example}
\begin{example}
  \begin{itemize}
  \item p.68
  \item フレーズ: 〜ゑーさに
  \item \ruby{素手}{んなでぃー}さーねー\ruby{個所}{ゆす}ねー\ruby{行}{い}からんゑーさに、\ruby{何}{ぬー}がな\ruby{買}{こー}てぃくーわ。
  \item 手ぶらでは他所に行けないでしょう、何か買っておいで。  
  \end{itemize}
\end{example}
\begin{example}
  \begin{itemize}
  \item p.40
  \item フレーズ: でーやー
  \item \ruby{飽}{に}りーくさりーさっとーる\ruby{人}{ちゅ}んでぃしぇー、\ruby{余程}{ゆふどぅ}ぬ\ruby{者}{むん}やん\highlight{でーやー}。
  \item 飽き果てられる人は、余程の者なんだろうね。
  \end{itemize}
\end{example}
\begin{example}
  \begin{itemize}
  \item p.26
  \item フレーズ: でーやー
  \item \ruby{大工}{せーく}や\ruby{道具勝}{どうぐまさゐ}んでぃ\ruby{昔}{んかし}から あん\ruby{言}{いゃ}っとーしが、あんどぅやん\highlight{でーやー}。
  \item 大工は道具勝と、昔から言われているようだが、その通りだろうね。
  \end{itemize}
\end{example}
\begin{example}
  \begin{itemize}
  \item p.40
  \item フレーズ: V$_{てぃ}$ + まー(に)
  \item \ruby{何}{ぬー}んでぃち\ruby{斯}{か}んなたが、\ruby{胸底}{んにすく}\ruby{落}{うとぅ}ち ゆー\ruby{考}{かんげー}てぃ\highlight{まー}。
  \item どうして斯うなったか、胸に手をあてて、よく考えてごらん。
  \end{itemize}
\end{example}
\begin{example}
  \begin{itemize}
  \item p.26
  \item フレーズ:  V$_{てぃ}$ + まー(に)
  \item (沖) \ruby{片付}{しぇーき}てぃ\highlight{まー}んでぃ\ruby{言}{いゃ}ってぃん、\ruby{食}{か}まりーる\ruby{丈}{うっさ}どぅ\ruby{食}{か}まりーる\ruby{押付}{うしち}きらんけー。
  \item (日) 全部、平らげてよと言われても、食べられるにも程がある。おしつけるなよ。
  \end{itemize}
\end{example}
\begin{example}
  \begin{itemize}
  \item p.26
  \item フレーズ: トーン形。変化動詞。完了相?
  \item (沖) \ruby{粗悪物}{そーべーむん}でぃる\ruby{沖縄語}{うちなーぐちぇー}、\ruby{商売物}{そーべーむん}からどぅ\ruby{転化}{かわ}てぃ\highlight{\ruby{来}{ちょー}る}ぎさーやんでぃどー。
  \item (日) 粗悪物は商売物からの転化した沖縄語らしいよ。
  \end{itemize}
\end{example}
\begin{example}
  \begin{itemize}
  \item p.95
  \item フレーズ: ビーン体のエー形(已然形)
  \item コメント: ~びーれー=[直訳]ますれば。順接の恒常条件。~と決まって、〜ときはいつも(学研全訳古語辞典)。
  \item (沖) \ruby{今日}{ちゅー}ぬ\ruby{商売}{あちねー}ぬ\ruby{口開}{みーぐち}\highlight{なとーゐびーれー} \ruby{安}{やし}みやびららんくとぅ \ruby{添分}{しーぶん}っし うさぎーさ。
  \item (日) 今日の商いの口開け\highlight{ですから}、安くすることはできませんが、おまけを添えてあげますよ。
  \end{itemize}
\end{example}
\begin{example}
  \begin{itemize}
  \item p.96
  \item フレーズ: 受け身の用法
  \item コメント: なさっとーん=なされた $\rightarrow$ 生んでくれた、生んでもらった。ナス:ナサリーン $\leftrightarrow$ うむ:うまれる(現代は自動詞だが形としては、「うむ」の受け身であることと関係がありそう)
  \item (沖) くれー\ruby{何}{ぬー}がな\ruby{可笑}{をぅかー}さいびーんやー、\ruby{男性}{をぃきが}やてぃん\ruby{女性}{をぃなぐ}んかいどぅ\ruby{産}{な}さっとーるむんぬやーさい。
  \item (日) これは何か変ですな、男だって女に生んでもらったんですけどね。
  \end{itemize}
\end{example}

%-----------  Bibliography ----------------
\begin{thebibliography}{10}
\bibitem{ハイス}
  ファン・デル・ルベ ハイス(2018)「沖縄語久米島謝名堂方言のテンス・アスペクト・エヴィデンシャリティー形式」法政大学沖縄文化研究所出版『琉球の方言』
\end{thebibliography}

\end{document}